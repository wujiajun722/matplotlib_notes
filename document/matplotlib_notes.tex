\documentclass[15pt,twocolumn]{ctexart}

\usepackage{amsmath, amsthm, amssymb, appendix, bm, graphicx, mathrsfs}
\usepackage{verbatim}
\usepackage[hidelinks]{hyperref}
\usepackage{listings} 
\usepackage{xcolor} 
\usepackage{subfigure}				%同时多个插图
\usepackage{fontspec}
\usepackage{asymptote}
\usepackage{draftwatermark}         % 所有页加水印
\usepackage{geometry}
\usepackage{media9}
\usepackage{multimedia}
\geometry{a4paper,centering,scale=0.8}
\usepackage{float}
%\usepackage[firstpage]{draftwatermark} % 只有第一页加水印
\SetWatermarkText{$W^uJ^{ia}J^{un}$}           % 设置水印内容
%\SetWatermarkText{\includegraphics{fig/texlion.png}}         % 设置水印logo
\SetWatermarkLightness{0.9}             % 设置水印透明度 0-1
\SetWatermarkScale{1} 
\usepackage{pifont}					%插入带有圈的数字例如\ding{172}
\usepackage{caption}				%插入无编号的图
\usepackage{fancyhdr} % 引入fancyhdr宏包


% 按章节重置图片编号(如图1.1)
\counterwithin{figure}{section}
\pagestyle{fancy} % 设置页面风格为fancy
\fancyhf{} % 清除默认页眉页脚设置
\fancyhead[L]{$\alpha \beta \gamma \delta \epsilon $\\$\rho \sigma \phi \psi \omega $} % 在页眉左边设置文本
\fancyhead[C]{\href{https://gitee.com/Wjj0123/matplotlib_notes}{中页眉}} % 在页眉中间设置文本
\fancyhead[R]{$matplotlib\  notes$} % 在页眉右边设置文本
\fancyfoot[C]{\thepage} % 在页脚中间设置页码
\renewcommand{\headrulewidth}{0.4pt} % 页眉下方的横线粗细
\renewcommand{\footrulewidth}{0pt} % 页脚横线粗细(这里设置为0pt表示没有横线)
% 全局设置
\lstdefinestyle{myStyle}{
	language=Python,
	basicstyle=\ttfamily, % 基本代码样式
	keywordstyle=\color{blue},  % 关键字样式
	stringstyle=\color{red},   % 字符串样式
	commentstyle=\color{gray}, % 注释样式
	morecomment=[l][\color{magenta}]{\#}, % Python注释样式
	basicstyle={\tiny}, % 基本代码字体大小
	numbers=left, % 显示行号在左侧
	numbersep=5pt, % 行号与文本之间的间距
	frame=single, % 单边框
	framerule=0.4pt, % 边框线条宽度
	rulecolor=\color{black}, % 边框线条颜色
	xleftmargin=1em, % 左边空白区域大小
	breaklines=true, % 长行换行
	showspaces=false, % 不显示空格符
	showtabs=false, % 不显示制表符
	tabsize=2, % 制表符占位数
	backgroundcolor=\color{white!50}, % 背景颜色
	aboveskip=3pt, % 代码块上方间距
	belowskip=3pt,	 % 代码块下方间距
	columns=fullflexible, % 字符间距
	linewidth=1\linewidth, % 代码块宽度
}

\lstset{
	style=myStyle,
}

\hypersetup{
	hidelinks,
	colorlinks = true,
	allcolors=black,
	pdfstartview=Fit,
	breaklinks=true
}


\ctexset{
	%chapter = {name = {第,章},number = \chinese{chapter}},
	section = {name= {\S,、},number = \arabic{section}},
	subsection = {name = {,、},number = \arabic{subsection}},
	subsubsection = {name = {,、},number = \arabic{subsection}.\arabic{subsubsection}}
}
\title{\textbf{\huge matplotlib\\笔记本}}
\author{吴嘉军}
\date{\today}
\linespread{1.4}
\newtheorem{theorem}{定理}[section]
\newtheorem{definition}[theorem]{定义}
\newtheorem{lemma}[theorem]{引理}
\newtheorem{corollary}[theorem]{推论}
\newtheorem{example}[theorem]{例}
\newtheorem{proposition}[theorem]{命题}
\renewcommand{\abstractname}{\Large\textbf{摘要}}


\begin{document}
	\setcounter{page}{0}
	\maketitle
	\thispagestyle{empty}
		\begin{abstract}
		这里是摘要
		\par\textbf{关键词:}这里是关键词1、关键词2
	\end{abstract}
	\newpage
	\pagenumbering{Roman}
	\setcounter{page}{1}
	\tableofcontents
	\newpage
	\setcounter{page}{1}
	\pagenumbering{arabic}
	
	\section{简单入门}
	一般来说,matplotlib需要和numpy,搭配着使用,因为matplotlib主要是用来绘图的,而numpy主要是用来进行数值计算的,两者结合起来可以更好地进行数据处理和可视化.
	\subsection{安装}
	在使用matplotlib之前,需要先安装它.可以使用pip命令进行安装
	\lstinputlisting{../code/section0/0.1.txt}
	安装完成后,可以使用以下命令来验证是否安装成功.

	\subsubsection{导入库}
	在使用matplotlib之前,需要先导入它.通常情况下,我们会导入matplotlib.pyplot模块,并将其重命名为plt.这样可以方便我们后续的使用.
	\lstinputlisting{../code/section0/0.2.txt}
	\subsubsection{创建数据}
	在绘图之前,我们需要先创建一些数据.通常情况下,我们会使用numpy来创建数据.例如,我们可以使用numpy的linspace函数来创建一个等间隔的数组.
	\lstinputlisting{../code/section0/0.3.txt}
	\subsection{绘制图形}
	使用matplotlib绘制图形非常简单.我们只需要调用plt.plot函数,并传入x和y数据即可.然后,使用plt.show函数来显示图形.
	\lstinputlisting{../code/section0/0.4.txt}

	下面是一个简单的例子,展示如何使用matplotlib绘制一个正弦波.
	\lstinputlisting[language=python]{../code/section1/1.1.py}
		
	运行上述代码,将会显示一个包含正弦波的图形窗口.
	\begin{figure}[H]
		\centering
		\includegraphics[width=0.8\linewidth]{../figures/section1/1.1.png}
		\caption{正弦波图形}
		\label{fig:sin_wave}
	\end{figure}
	\subsection{插入中文}
	在使用matplotlib绘图时,如果需要插入中文,可以使用以下方法:
	\begin{lstlisting}
	plt.xlabel('X轴标签', fontproperties='SimHei')  # 设置X轴标签为中文
	plt.ylabel('Y轴标签', fontproperties='SimHei')  # 设置Y轴标签为中文
	plt.title('图形标题', fontproperties='SimHei')  # 设置图形标题为中文
	\end{lstlisting}
	这里的'SimHei'是黑体字体的名称,你也可以根据需要选择其他字体.确保你的系统中已经安装了所需的中文字体.
	\lstinputlisting[language=python]{../code/section1/1.2.py}
	\begin{figure}[H]
		\centering
		\includegraphics[width=0.8\linewidth]{../figures/section1/1.2.png}
		\caption{}
		\label{fig:sin_wave_chinese}
	\end{figure}
	如果觉得太麻烦了,可以在代码开头添加如下代码,这样就可以全局使用中文字体了.
	\lstinputlisting{../code/section0/0.5.txt}
	运行上述代码,将可以在fig中添加中文,以下代码为例:
	\lstinputlisting[language=python]{../code/section1/1.3.py}

	\subsection{坐标布局}
	在matplotlib中,可以通过多种方式来调整图形的坐标布局.以下是一些常用的方法:
	\lstinputlisting[language=python]{../code/section1/1.4.py}
	运行上述代码,将会显示一个包含三个子图的图形窗口,一个在左,两个在右.如下所示:
	\begin{figure}[H]
		\centering
		\includegraphics[width=0.8\linewidth]{../figures/section1/1.4.png}
		\caption{多个子图布局}
		\label{fig:subplots}
	\end{figure}

	\subsection{坐标轴}
	在matplotlib中,可以通过多种方式来调整图形的坐标轴.以下是一些常用的方法:
	\lstinputlisting[language=python]{../code/section1/1.5.py}
	运行上述代码,将会显示一个包含自定义坐标轴的图形窗口.如下所示:
	\begin{figure}[H]
		\centering
		\includegraphics[width=0.8\linewidth]{../figures/section1/1.5.png}
		\caption{自定义坐标轴}
		\label{fig:custom_axes}
	\end{figure}

	\subsection{箭头}
	在matplotlib中,可以通过多种方式来添加箭头.以下是一些常用的方法:
	\lstinputlisting[language=python]{../code/section1/1.6.py}
	运行上述代码,将会显示一个包含箭头的图形窗口.如下所示:
	\begin{figure}[H]
		\centering
		\includegraphics[width=0.8\linewidth]{../figures/section1/1.6.png}
		\caption{箭头示例}
		\label{fig:箭头实例}
	\end{figure}
	再举例说明一下如何添加不同样式的箭头.
	\lstinputlisting[language=python]{../code/section1/1.7.py}
	运行上述代码,将会显示一个包含不同样式箭头的图形窗口.如下所示:
	\begin{figure}[H]
		\centering
		\includegraphics[width=0.8\linewidth]{../figures/section1/1.7.png}
		\caption{不同样式箭头示例}
		\label{fig:不同样式箭头}
	\end{figure}
	\newpage
	%参考文献

	\subsection{文字说明}
	在matplotlib中,可以通过多种方式来添加文字说明.以下是一些常用的方法:
	\lstinputlisting[language=python]{../code/section1/1.8.py}
	运行上述代码,将会显示一个包含文字说明的图形窗口.如下所示:
	\begin{figure}[H]		
		\centering
		\includegraphics[width=0.8\linewidth]{../figures/section1/1.8.png}
		\caption{文字说明示例}
		\label{fig:文字说明示例}
	\end{figure}

	\newpage
	\section{patches}
	在matplotlib中,patches模块提供了许多用于绘制图形元素的类.这些类可以用来创建各种形状,如矩形、圆形、椭圆等.以下是一些常用的patches类:
	\begin{itemize}
		\item \texttt{Rectangle}: 用于绘制矩形.
		\item \texttt{Circle}: 用于绘制圆形.
		\item \texttt{Ellipse}: 用于绘制椭圆.
		\item \texttt{Polygon}: 用于绘制多边形.
	\end{itemize}
	
	例如,我们可以使用patches.Rectangle来创建一个正方形:
	\lstinputlisting[language=python]{../code/section2/2.1.py}
	
	运行上述代码,将会显示一个包含正方形的图形窗口.如下所示:
	\begin{figure}[H]
		\centering
		\includegraphics[width=0.8\linewidth]{../figures/section2/2.1.png}
		\caption{正方形示例}
		\label{fig:rectangle_example}
	\end{figure}
	\subsection{矩形}
	上面的例子已经展示了如何使用$patches.Rectangle$来创建一个矩形.我们还可以通过设置不同的参数来调整矩形的属性,如颜色、边框宽度等.
	\lstinputlisting[language=python]{../code/section2/2.2.py}
	运行上述代码,将会显示一个包含自定义矩形的图形窗口.如下所示:
	\begin{figure}[H]
		\centering
		\includegraphics[width=0.8\linewidth]{../figures/section2/2.2.png}
		\caption{自定义矩形示例}
		\label{fig:custom_rectangle}
	\end{figure}
	
	\subsection{三角形}
	我们可以使用$patches.Polygon$来创建一个三角形.以下是一个示例:
	\lstinputlisting[language=python]{../code/section2/2.3.py}
	运行上述代码,将会显示一个包含三角形的图形窗口.如下所示:
	\begin{figure}[H]
		\centering
		\includegraphics[width=0.8\linewidth]{../figures/section2/2.3.png}
		\caption{三角形示例}
		\label{fig:triangle_example}
	\end{figure}
	这样绘制的三角形是已知顶点坐标的.如果是只知道两条边长和其中的夹角,可以通过计算得到其他两个顶点的坐标.
	\subsubsection{边角边画三角形}
	举个例子,假设已知三角形的一条边长为$a$,另一条边长为$b$,夹角为$\theta$,则可以通过以下代码计算出其他两个顶点的坐标并绘制三角形:
	\lstinputlisting[language=python]{../code/section2/2.4.py}
	\begin{figure}[H]
		\centering
		\includegraphics[width=0.8\linewidth]{../figures/section2/2.4.png}
		\caption{三角形示例2}
		\label{fig:triangle_example2}
	\end{figure}
	\subsubsection{向量}
	还可以使用向量在图像中需找特殊点,测量边长,测量角,绘制角平分线.
	\lstinputlisting[language=python]{../code/section2/2.5.py}
	\begin{figure}[H]
		\centering
		\includegraphics[width=0.8\linewidth]{../figures/section2/2.5.png}
		\caption{三角形示例3}
		\label{fig:triangle_example3}
	\end{figure}
	\subsubsection{三边画三角形}
	如果知道三条边的长度a,b,c,也可以用于绘制三角形.
	\lstinputlisting[language=python]{../code/section2/2.6.py}
	\begin{figure}[H]
		\centering
		\includegraphics[width=0.8\linewidth]{../figures/section2/2.6.png}
		\caption{三角形示例4}
		\label{fig:triangle_example4}
	\end{figure}

	\subsubsection{角边角画三角形}
	知道三角形的两个角,和两角中间的边长,也可以绘制三角形.可以运用正弦定理:
	
	$\frac{a}{\sin A} = \frac{b}{\sin B} = \frac{c}{\sin C}$
	用下面的例子绘图:
	\lstinputlisting[language=python]{../code/section2/2.7.py}
	\begin{figure}[H]
		\centering
		\includegraphics[width=0.8\linewidth]{../figures/section2/2.7.png}
		\caption{三角形示例5}
		\label{fig:triangle_example5}
	\end{figure}

	\subsection{角}
	\lstinputlisting[language=python]{../code/section2/2.8.py}
	\begin{figure}[H]
		\centering
		\includegraphics[width=0.8\linewidth]{../figures/section2/2.8.png}
		\caption{角}
		\label{fig:angle}
	\end{figure}

	\subsection{圆形和椭圆}
	\lstinputlisting[language=python]{../code/section2/2.9.py}
	\begin{figure}[H]
		\centering
		\includegraphics[width=0.8\linewidth]{../figures/section2/2.9.png}
		\caption{圆形和椭圆}
		\label{fig:angle}
	\end{figure}
	\subsection{扇形}
	\lstinputlisting[language=python]{../code/section2/2.10.py}
	\begin{figure}[H]
		\centering
		\includegraphics[width=0.8\linewidth]{../figures/section2/2.10.png}
		\caption{扇形}
		\label{fig:angle}
	\end{figure}
	plt也可以绘制扇形饼图
	\lstinputlisting[language=python]{../code/section2/2.11.py}
	\begin{figure}[H]
		\centering
		\includegraphics[width=0.8\linewidth]{../figures/section2/2.11.png}
		\caption{扇形饼图}
		\label{fig:angle}
	\end{figure}
	\section{隐函数}
	matplotlib也可以按照隐函数进行绘图.
	这里主要提供两种方法进行画图
	\subsection{圆锥曲线标准方程}
	
	$\frac{x^2}{a^2} + \frac{y^2}{b^2} = 1$
	\lstinputlisting[language=python]{../code/section3/3.1.py}
	\begin{figure}[H]
		\centering
		\includegraphics[width=0.8\linewidth]{../figures/section3/3.1.png}
		\caption{椭圆}
		\label{fig:example}
	\end{figure}
	\begin{figure}[H]
		\centering
		\includegraphics[width=0.8\linewidth]{../figures/section3/3.2.png}
		\caption{双曲线}
		\label{fig:example}
	\end{figure}
	\begin{figure}[H]
		\centering
		\includegraphics[width=0.8\linewidth]{../figures/section3/3.3.png}
		\caption{抛物线}
		\label{fig:example}
	\end{figure}
	以上可以看出画图的方式是用到了contour绘制等高线的方式.下面我们还可以用到sympy库,利用
	sympy可以直接解出y与x的关系表达式:$y = f(x)$
	\subsection{圆锥曲线一般方程}
	圆锥曲线一般方程:
	\begin{align*}
		Ax^2 + Bxy + Cy^2 + Dx + Ey + F = 0
	\end{align*}
	用圆锥曲线的一般方程来绘制一些图片
	\lstinputlisting[language=python]{../code/section3/3.5.py}
	\begin{figure}[H]
		\centering
		\includegraphics[width=0.8\linewidth]{../figures/section3/3.5.png}
		\caption{圆锥曲线一般方程}
		\label{fig:example}
	\end{figure}
	

	\section{sympy库}
sympy是python的一个库,可以帮忙带着符号进行运算.下面用几个例子来说明.
    \subsection{sympy解方程}
    \lstinputlisting[language=python]{../code/section4/4.1.py}
	\begin{figure}[H]
		\centering
		\includegraphics[width=0.8\linewidth]{../figures/section4/4.1.png}
		\caption{解方程}
		\label{fig:example}
	\end{figure}
    \subsection{sympy解二元方程}
    \lstinputlisting[language=python]{../code/section4/4.2.py}
	\begin{figure}[H]
		\centering
		\includegraphics[width=0.8\linewidth]{../figures/section4/4.2.png}
		\caption{解方程}
		\label{fig:example}
	\end{figure}
    \subsection{含参方程}
    例如有一个方程,ax + b = 0,求解出x的表达式:
    \begin{align*}
        x = \frac{-b}{a}
    \end{align*}
    利用sympy也可以方便的求解.
    \lstinputlisting[language=python]{../code/section4/4.3.py}
	一元二次方程$ax^2 + bx + c = 0$也可以求解
	\lstinputlisting[language=python]{../code/section4/4.4.py}
	\begin{figure}[H]
		\centering
		\includegraphics[width=0.8\linewidth]{../figures/section4/4.4.png}
		\caption{含参方程}
		\label{fig:example}
	\end{figure}

	




	\newpage
	\begin{thebibliography}{99}
		\bibitem{matplotlib} Matplotlib官方文档, \url{https://matplotlib.org/stable/contents.html}
		\bibitem{numpy} NumPy官方文档, \url{https://numpy.org/doc/}
	\end{thebibliography}	
\end{document}