\section{简单入门}
	一般来说,matplotlib需要和numpy,搭配着使用,因为matplotlib主要是用来绘图的,而numpy主要是用来进行数值计算的,两者结合起来可以更好地进行数据处理和可视化.
	\subsection{安装}
	在使用matplotlib之前,需要先安装它.可以使用pip命令进行安装
	\lstinputlisting{../code/section0/0.1.txt}
	安装完成后,可以使用以下命令来验证是否安装成功.

	\subsubsection{导入库}
	在使用matplotlib之前,需要先导入它.通常情况下,我们会导入matplotlib.pyplot模块,并将其重命名为plt.这样可以方便我们后续的使用.
	\lstinputlisting{../code/section0/0.2.txt}
	\subsubsection{创建数据}
	在绘图之前,我们需要先创建一些数据.通常情况下,我们会使用numpy来创建数据.例如,我们可以使用numpy的linspace函数来创建一个等间隔的数组.
	\lstinputlisting{../code/section0/0.3.txt}
	\subsection{绘制图形}
	使用matplotlib绘制图形非常简单.我们只需要调用plt.plot函数,并传入x和y数据即可.然后,使用plt.show函数来显示图形.
	\lstinputlisting{../code/section0/0.4.txt}

	下面是一个简单的例子,展示如何使用matplotlib绘制一个正弦波.
	\lstinputlisting[language=python]{../code/section1/1.1.py}
		
	运行上述代码,将会显示一个包含正弦波的图形窗口.
	\begin{figure}[H]
		\centering
		\includegraphics[width=0.8\linewidth]{../figures/section1/1.1.png}
		\caption{正弦波图形}
		\label{fig:sin_wave}
	\end{figure}
	\subsection{插入中文}
	在使用matplotlib绘图时,如果需要插入中文,可以使用以下方法:
	\begin{lstlisting}
	plt.xlabel('X轴标签', fontproperties='SimHei')  # 设置X轴标签为中文
	plt.ylabel('Y轴标签', fontproperties='SimHei')  # 设置Y轴标签为中文
	plt.title('图形标题', fontproperties='SimHei')  # 设置图形标题为中文
	\end{lstlisting}
	这里的'SimHei'是黑体字体的名称,你也可以根据需要选择其他字体.确保你的系统中已经安装了所需的中文字体.
	\lstinputlisting[language=python]{../code/section1/1.2.py}
	\begin{figure}[H]
		\centering
		\includegraphics[width=0.8\linewidth]{../figures/section1/1.2.png}
		\caption{}
		\label{fig:sin_wave_chinese}
	\end{figure}
	如果觉得太麻烦了,可以在代码开头添加如下代码,这样就可以全局使用中文字体了.
	\lstinputlisting{../code/section0/0.5.txt}
	运行上述代码,将可以在fig中添加中文,以下代码为例:
	\lstinputlisting[language=python]{../code/section1/1.3.py}

	\subsection{坐标布局}
	在matplotlib中,可以通过多种方式来调整图形的坐标布局.以下是一些常用的方法:
	\lstinputlisting[language=python]{../code/section1/1.4.py}
	运行上述代码,将会显示一个包含三个子图的图形窗口,一个在左,两个在右.如下所示:
	\begin{figure}[H]
		\centering
		\includegraphics[width=0.8\linewidth]{../figures/section1/1.4.png}
		\caption{多个子图布局}
		\label{fig:subplots}
	\end{figure}

	\subsection{坐标轴}
	在matplotlib中,可以通过多种方式来调整图形的坐标轴.以下是一些常用的方法:
	\lstinputlisting[language=python]{../code/section1/1.5.py}
	运行上述代码,将会显示一个包含自定义坐标轴的图形窗口.如下所示:
	\begin{figure}[H]
		\centering
		\includegraphics[width=0.8\linewidth]{../figures/section1/1.5.png}
		\caption{自定义坐标轴}
		\label{fig:custom_axes}
	\end{figure}

	\subsection{箭头}
	在matplotlib中,可以通过多种方式来添加箭头.以下是一些常用的方法:
	\lstinputlisting[language=python]{../code/section1/1.6.py}
	运行上述代码,将会显示一个包含箭头的图形窗口.如下所示:
	\begin{figure}[H]
		\centering
		\includegraphics[width=0.8\linewidth]{../figures/section1/1.6.png}
		\caption{箭头示例}
		\label{fig:箭头实例}
	\end{figure}
	再举例说明一下如何添加不同样式的箭头.
	\lstinputlisting[language=python]{../code/section1/1.7.py}
	运行上述代码,将会显示一个包含不同样式箭头的图形窗口.如下所示:
	\begin{figure}[H]
		\centering
		\includegraphics[width=0.8\linewidth]{../figures/section1/1.7.png}
		\caption{不同样式箭头示例}
		\label{fig:不同样式箭头}
	\end{figure}
	\newpage
	%参考文献

	\subsection{文字说明}
	在matplotlib中,可以通过多种方式来添加文字说明.以下是一些常用的方法:
	\lstinputlisting[language=python]{../code/section1/1.8.py}
	运行上述代码,将会显示一个包含文字说明的图形窗口.如下所示:
	\begin{figure}[H]		
		\centering
		\includegraphics[width=0.8\linewidth]{../figures/section1/1.8.png}
		\caption{文字说明示例}
		\label{fig:文字说明示例}
	\end{figure}

	\newpage