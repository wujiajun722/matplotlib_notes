\section{隐函数}
	matplotlib也可以按照隐函数进行绘图.
	这里主要提供两种方法进行画图
	\subsection{圆锥曲线标准方程}
	
	$\frac{x^2}{a^2} + \frac{y^2}{b^2} = 1$
	\lstinputlisting[language=python]{../code/section3/3.1.py}
	\begin{figure}[H]
		\centering
		\includegraphics[width=0.8\linewidth]{../figures/section3/3.1.png}
		\caption{椭圆}
		\label{fig:example}
	\end{figure}
	\begin{figure}[H]
		\centering
		\includegraphics[width=0.8\linewidth]{../figures/section3/3.2.png}
		\caption{双曲线}
		\label{fig:example}
	\end{figure}
	\begin{figure}[H]
		\centering
		\includegraphics[width=0.8\linewidth]{../figures/section3/3.3.png}
		\caption{抛物线}
		\label{fig:example}
	\end{figure}
	以上可以看出画图的方式是用到了contour绘制等高线的方式.下面我们还可以用到sympy库,利用
	sympy可以直接解出y与x的关系表达式:$y = f(x)$
	\subsection{圆锥曲线一般方程}
	圆锥曲线一般方程:
	\begin{align*}
		Ax^2 + Bxy + Cy^2 + Dx + Ey + F = 0
	\end{align*}
	用圆锥曲线的一般方程来绘制一些图片
	\lstinputlisting[language=python]{../code/section3/3.5.py}
	\begin{figure}[H]
		\centering
		\includegraphics[width=0.8\linewidth]{../figures/section3/3.5.png}
		\caption{圆锥曲线一般方程}
		\label{fig:example}
	\end{figure}
	
